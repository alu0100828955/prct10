\documentclass[spanish,a4paper,10pt]{article}

\usepackage{latexsym,amsfonts,amssymb,amstext,amsthm,float,amsmath}
\usepackage[spanish]{babel}
\usepackage[latin1]{inputenc}
\usepackage[dvips]{epsfig}
\usepackage{graphicx}

\begin{document}
\title{Aproximacion del numero $\pi$ con una maquina de computo}
\author{Carlos Herrera Carballo \\ Practica de Laboratorio \#10}
\date{9 de abril de 2014}

\maketitle

\begin{abstract}
El objetivo de esta practica es entregar un informe escrito en \LaTeX{}
\end{abstract}
\section{Motivaci�n y Objetivos}
A lo largo de la historia han sido muchas las formas utilizadas por el
ser humano para calcular aproximaciones cada vez m�s exactas del n�mero $\pi$.
%
El objetivo de esta pr�ctica de laboratorio es implementar el c�digo \textsf{Python}
que permita a\-pro\-xi\-mar el n�mero $\pi$ con una cierta precisi�n.
%
$\pi$ se puede calcular mediante integraci�n:

$$\int_{0}^{1} \! \frac{4}{1+x^2}\, dx = 4(atan(1) -atan(0)) = \pi $$
\subsection{Informaci�n previa}
En el actual plan de estudios de Matem�ticas surge de la adaptaci�n de las titulaciones al Espacio Europeo de Educaci�n Superior. La principal caracter�stica es que est� basado en competencias. En la asignatura T�cnicas Experimentales se ha de desarrollar la competencia transversal:
Comunicar, tanto por escrito como de forma oral, conocimientos,procedimientos,resultados e ideas matem�ticas.
\section{Ejercicios propuestos}
\subsection{Descripci�n del ejercicio a entregar}
Se ha de crear un directorio en la carpeta de proyecto, en el que se han de guardar los archivos que se generen al realizar esta pr�ctica\footnote{Nota 1: Esta pr�ctica es la pr�ctica 10 y esto es un ejemplo de pie de p�gina}. Luego debe realizar un informe que contenga ciertos componentes b�sicos\footnote{Nota 2:Los componentes b�sicos est�n especificados en el gui�n de la pr�ctica, y esto es un segundo ejemplo de pie de p�gina}.

\section{Insertando un gr�fico}
\includegraphics[scale=0.15]{imagen1.eps}
\label{Mi primer grafico}

En \ref{Mi primer grafico} aparece el primer gr�fico de este informe.

\section{Insertando una tabla}
\begin{tabular}{lrc}
Nombre & Coche & Antig�edad \\
\hline
Pepe & seat & 3 \\
Juana & opel & 1 \\
Carlos & ferrari & 5
\label{Mi primera tabla}
\end{tabular}


En \ref{Mi primera tabla} aparece la primera tabla de este informe.

\end{document}